\vspace{-0.5cm}
\section*{Allgemein}
	\begin{minipage}[t]{12.5cm}
		\subsection*{Primitive Datentypen}
			\rowcolors{1}{gray!15}{white}
			\begin{tabular}{|>{\bfseries}l l l l|}
				\hline Typ & \bfseries{Beschreibung} & \bfseries{Wertebereich} & \bfseries{Wrapper-Klasse}
				\\\hline   boolean & Boolescher Wert & true, false & Boolean
				\\ char & Textzeichen (UTF16) & 'a', 'B', '0', 'é' etc. & Character
				\\ byte & Ganzzahl (8 Bit) & -128 bis 127 & Byte
				\\ short & Ganzzahl (16 Bit) & -32'768 bis 32'767 & Short
				\\ int & Ganzzahl (32 Bit) & -2$^{31}$ bis 2$^{31}$-1 & Integer
				\\ long & Ganzzahl (64 Bit) & -2$^{63}$ bis 2$^{63}$-1, 1L (L Suffix) & Long
				\\ float & Gleitkommazahl(32 Bit) & 0.1f, 2e4f (2*10$^4$) & Float
				\\ double & Gleitkommazahl(64 Bit) & 0.1, 2e4 & Double
				\\\hline
			\end{tabular}
		\end{minipage}
		\hspace*{0.6cm}
		\begin{minipage}[t]{5.7cm}
			\vspace*{0.7cm}
			\textbf{Überlauf bzw. Unterlauf} ist in Java definiert. Bei einem Überlauf wird ganz unten weitergezählt, bei einem Unterlauf wird von ganz oben fortgesetzt. Bei Gleitkommazahlen wird führt ein Über-/Unterlauf zu POSITIVE\_INFINITY bzw. NEGATIVE\_INFINITY.
			\subsection*{Explizite Typkonversation}
			Nur C-Style Cast: (int)3.5; $\rightarrow$ 3
			\\\todo{evtl. noch Arrays und Mehrdimensionale Arrays}
		\end{minipage}
	
	\vspace{0.3cm}
	\todo{evtl. "var" beschreiben(folie 7, Woche 4)}