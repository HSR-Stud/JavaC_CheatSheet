\vspace{-0.5cm}
\section*{Allgemein}
	\begin{minipage}[t]{10cm}
		\subsection*{Primitive Datentypen}
			\rowcolors{1}{gray!15}{white}
			\begin{tabular}{|>{\bfseries}l l l|}
				\hline   boolean & Boolescher Wert & true, false
				\\\hline char & Textzeichen (UTF16) & 'a', 'B', '0', 'é' etc.
				\\\hline byte & Ganzzahl (8 Bit) & -128 bis 127
				\\\hline short & Ganzzahl (16 Bit) & -32'768 bis 32'767
				\\\hline int & Ganzzahl (32 Bit) & -2$^{31}$ bis 2$^{31}$-1
				\\\hline long & Ganzzahl (64 Bit) & -2$^{63}$ bis 2$^{63}$-1, 1L (L Suffix)
				\\\hline float & Gleitkommazahl(32 Bit) & 0.1f, 2e4f (2*10$^4$)
				\\\hline double & Gleitkommazahl(64 Bit) & 0.1, 2e4 
				\\\hline
			\end{tabular}
		\end{minipage}
		\hspace*{0.6cm}
		\begin{minipage}[t]{8.4cm}
			\vspace*{0.7cm}
			\textbf{Überlauf/Unterlauf} ist in Java definiert. Zählt einfach, je nach dem, unten oder oben weiter. Bei Gleitkommazahlen wird 2*1e308 zu POSITIVE\_INFINITY.
			\subsection*{Explizite Typkonversation}
			Nur C-Style Cast: (int)3.5; $\rightarrow$ 3
			\\\todo{evtl. noch Arrays und Mehrdimensionale Arrays}
		\end{minipage}
	
	\vspace{0.3cm}\todo{evtl. komplexe Datentypen}
	\vspace{0.3cm}
	\lstinputlisting{code/test.java}