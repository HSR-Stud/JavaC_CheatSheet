\section*{Lambdas}
	\subsection*{Höherwertige Funktionen}
		\begin{itemize}[noitemsep]
			\item Funktionen, welche wiederum Funktionen als Parameter erwarten oder zurückgeben
			\item Funktionen werden wie Werte behandelt. (Referenz auf Funktion)
			\item Parametertyp ist ein Interface mit \textbf{genau einer} abstrakten Methode (\textbf{functional Interface})
			\item übergebene Methode muss \textbf{typ-kompatibel} mit der Methode im Interface sein, sprich gleiche Signatur und Rückgabetyp
		\end{itemize}
	
	\subsection*{Methodenreferenz}
		\begin{minipage}{6.3cm}
			\begin{itemize}[noitemsep]
				\item Referenz auf eine Methode (noch kein Aufruf!)
				\item Methode wird wie ein Objekt behandelt
				\item Methodenreferenzen haben keinen eigenen Typ (nicht wie in \texttt{C++})
			\end{itemize}
		\end{minipage}
		\hspace*{0.5cm}
		\begin{minipage}{12cm}
			\lstinputlisting{code/Lambdas_FunctionRef.java}
		\end{minipage}
		Syntax einiger nützlichen Methodenreferenzen:
		\lstinputlisting{code/Lambdas_FunctionRefCalls.java}
	\subsection*{Lambdas}
		Ein Lambda ist eine Referenz auf eine \textbf{anonyme} Methode. Der Syntax sieht wie folgt aus:
		\lstinputlisting{code/Lambdas_Syntax.java}
		Beispiel:
		\lstinputlisting{code/Lambdas_Syntax2.java}
		Dabei sind die geschweiften Klammern im Body optional, werden nur bei einem syntaktischer Ausdruck (Expression) gebraucht.
		\lstinputlisting{code/Lambdas_Syntax3.java}
		Bei nur einem Übergabeparameter sind sogar die runden Klammern überflüssig.
		\lstinputlisting{code/Lambdas_Syntax4.java}
		letzterer Ausdruck ist Analog zu \texttt{\textbf{this::isAdult}} mit:
		\lstinputlisting{code/Lambdas_Syntax4_analog.java}
		Lambdas sind auch ohne Parameterliste möglich, was dann wie folgt aussieht:
		\lstinputlisting{code/Lambdas_Syntax5.java}
		Das benützen eines Lambdas und das Functional Inteface sieht dann wie folgt aus:\\
		\begin{minipage}[t]{11.3cm}
			\lstinputlisting{code/Lambdas_usage.java}
		\end{minipage}
		\hspace*{0.5cm}
		\begin{minipage}[t]{7cm}
			\lstinputlisting{code/Lambdas_FunctionalInterface.java}
			\texttt{@FunctionalInterface} ist nicht nötig, aber kann zur Information hingeschrieben werden. Für den Compiler ändert sicher aber nichts.
		\end{minipage}